%Preamble
\documentclass[12pt]{article}
\usepackage[utf8]{inputenc}
\usepackage{graphicx}
\usepackage{setspace}
\setstretch{1.5}
\usepackage{geometry}
\geometry{a4paper, portrait, margin=30mm, bmargin=30mm, tmargin=30mm}
\usepackage{amsmath}

%Bibliography
%\usepackage[backend = biber, style = alphabetic, sorting = ynt]{biblatex}
%\addbibresource{thesis.bib}
\bibliographystyle{unsrt}

%Patch abstract command to make it normal size
\makeatletter
\renewenvironment{abstract}{%
    \if@twocolumn
      \section*{\abstractname}%
    \else %% <- here I've removed \small
      \begin{center}%
        {\bfseries \Large\abstractname\vspace{\z@}}%  %% <- here I've added \Large
      \end{center}%
      \quotation
    \fi}
    {\if@twocolumn\else\endquotation\fi}
\makeatother


\begin{document}
\begin{center}
\thispagestyle{empty}
\large{\textbf{Univserity of Innsbruck}}\\[-0.9ex]
\large{Department of General, Inorganic and Theoretical Chemistry}\\
\vspace{0.3cm}
\begin{center}
\includegraphics[width=7cm]{Images/Logo.jpg}\\
\vspace{0.9cm}
\textbf{\LARGE{Master Thesis}}
\medskip\par
\vspace{1.2cm}
\Large{\textbf{Structure and Thermodynamics of \\ Guest@MOF Material}}\\[-0.5ex]
\vspace*{1.5cm}
\large{Investigating GUEST in MIL-68Ga via a Molecular Dynamics Simulation Approach}\\[-1.5ex]
\vspace*{1.5cm}
\bigskip\par
\textbf{Michael Helmut Fill, BSc. }\\[-1ex]
\medskip
\textbf{Supervisor:} Assoc. Prof. Dr. Thomas Hofer\\
Hier Datum einfügen
\end{center}
\end{center}

\newpage

\begin{abstract}
  This thesis is dedicated to the brave Mujahideen fighters of Afghanistan.
\end{abstract}

\newpage

\tableofcontents

\newpage

\section{Introduction}

\section{Theory}
\subsection{Quantum Chemistry}
At the dawn of the 20th century, general consensus among many physicists versed in classical mechanics was that the
fundamental laws of nature had been solved, and that the by then still unsolved problems of e.g.~black body radiation
would be resolved in due time. However, when Max Planck published his ideas on the quantization of energy in 1900~\cite{Planck1901}, and especially
after Albert Einstein realistically explained the photoelectric effect using Plancks' hypothesis in 1905~\cite{Einstein1905}, it became clear that the classical laws of physics were 
inadequate at explaining the behavior of matter on the atomic scale.
The field of quantum mechanics was born, and together with further groundbreaking contributions by, among others, Niels Bohr~\cite{Bohr1913}, Werner Heisenberg~\cite{Heisenberg1927} and Louis de Broglie~\cite{Broglie1924},
paved the way for Erwin Schrödinger to lay the groundwork for quantum chemistry with his wave equation in 1926~\cite{Schrdinger1926}\cite{Schrdinger1926-2}.
\subsubsection{The Schrödinger Equation}
Starting from the initial assumption that all properties of a given system could be described by a wave function $\Psi$, Schrödinger derived his now well-known
time-independant~(\ref{eq:schrodingerIndependant}) and, more general, time-dependant~(\ref{eq:schrodingerDependant}) equations.


\begin{equation}
  \hat{H}\Psi = E\Psi
  \label{eq:schrodingerIndependant}
\end{equation}
\begin{equation}
  i\hbar\frac{\partial}{\partial t}\Psi = \hat{H}\Psi
  \label{eq:schrodingerDependant}
\end{equation}

\noindent With $\Psi$ being the eigenfunction of the system, $\hbar$ the reduced Planck constant, $\hat{H}$ the Hamiltonian operator, $t$ the time and $E$ the energy of the system, or eigenvalue of the Hamiltonian operator.
The Hamiltonian operator~(\ref{eq:hamiltonian}) is defined as the sum of the kinetic and potential energy operators, which, when applied to a state function $\Psi$, yields the total energy of the system as its eigenvalue to the eigenfunction $\Psi$.

\begin{equation}
  \hat{H} = -\frac{\hbar^2}{2m}\nabla^2 + E_{pot}
  \label{eq:hamiltonian}
\end{equation}

\bigskip

\noindent Here, $\nabla^2$ is the Laplace operator, $m$ the mass of the particle and $E_{pot}$ the potential energy of the system.
For a single particle with mass $m$ moving in three-dimensional space, the time independant Schrödinger equation can then be written as~(\ref{eq:differential}).

\begin{equation}
  -\frac{\hbar^2}{2m}\nabla^2\Psi + E_{pot}\Psi = E\Psi
  \label{eq:differential}
\end{equation}

\begin{equation*}
  \nabla^2 = \frac{\partial^2}{\partial x^2} + \frac{\partial^2}{\partial y^2} + \frac{\partial^2}{\partial z^2}
\end{equation*}

\bigskip

\noindent This linear differential equation~(\ref{eq:differential}) forms the basis for determining the wave function of a given system. 
A wavefunction $\Psi$ that satisfies the Schrödinger equation and results in its respective eigenvalue $E$ must therefore be an eigenfunction to the Hamiltonian operator.
It is due to this that finding the correct wave function $\Psi$ for a system is referred to as an eigenvalue problem.\\
Hereby, it is important to note that a wavefunction is not directly observable and therefore no meaning can be attributed to its values.
However, the square modulus of the wavefunction $|\Psi|^2$, as interpreted by Max Born~\cite{Born1926}, is proportional to the observable probability density of an electron in the given system.

\subsection{Molecular Dynamics Simulations}
Alongside advancements in the field of digital electronics came the possibilities of simulating chemical 
systems on ever larger scales. Where practical experiments are limited by equipment, time, resources and safety regulations, 
simulations can be conducted cost effectively and with an ever increasing level of precision.
Simulating a chemical system to receive an ensemble of configurations starts with choosing the most viable
theoretical framework. Molecular Dynamics (MD), utilizing the DFTB method, is one of those frameworks particularly well suited for  
the theoretical time-dependant analysis of periodic chemical systems. 

\bibliography{thesis}

\end{document}